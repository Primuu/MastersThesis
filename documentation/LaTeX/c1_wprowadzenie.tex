\chapter{Wprowadzenie}
W literaturze przedmiotu analiza sentymentu i emocji definiowana jest jako dziedzina badań analizująca opinie, sentymenty, oceny, postawy i emocje ludzi wobec różnych podmiotów, takich jak osoby, organizacje, produkty, usługi czy wydarzenia~\cite{Liu2012}. Wymienione opinie często przekazywane są w formie pisanej, głównie za pośrednictwem platform internetowych, takich jak Facebook, X (dawniej Twitter), Reddit czy inne media społecznościowe, gdzie użytkownicy regularnie publikują komentarze i posty wyrażające ich zdanie na dany temat. Na przykład, według Internet Live Stats, dziennie na platformie X pojawia się 500 milionów komentarzy i wpisów~\cite{TwitterStats}. Ze względu na ogromną liczbę dostępnych tekstów, ręczna identyfikacja opinii w nich zawarta, jest zarówno trudna jak i czasochłonna, co uzasadnia potrzebę stosowania systemów zautomatyzowanych~\cite{Liu2012}.

W ramach szeroko rozumianej analizy sentymentu można wyróżnić dwa główne podejścia:
\begin{enumerate}
    \item Analiza sentymentu -- polega na klasyfikacji tekstu pod względem ogólnego wydźwięku na~kategorie takie jak pozytywny, negatywny czy neutralny. Wyzwania w tej dziedzinie NLP wynikają głównie z problemów kontekstowych oraz wieloznaczności językowej. Sarkazm lub~ironia mogą prowadzić do błędnych interpretacji, gdyż wydźwięk słów pozornie pozytywnych może mieć negatywne znaczenie, np. „Świetnie, uwielbiam stać w korkach!”~\cite{Sarcasm}.
    \item Analiza emocji -- koncentruje się na wykrywaniu bardziej szczegółowych stanów emocjonalnych, takich jak radość, smutek czy gniew. W tym wypadku również często wymagane jest zrozumienie kontekstu oraz niuansów językowych. Dodatkową trudnością może być współwystępowanie różnych emocji, co utrudnia jednoznaczną klasyfikację. Na przykład tekst może jednocześnie wyrażać strach i zaskoczenie, jak w zdaniu „Nie mogłem uwierzyć własnym oczom, kiedy nagle zobaczyłem tę dziwną postać -- przeszywający mnie strach sparaliżował mnie na chwilę”, lub smutek i gniew, co ilustruje wypowiedź: „Czuję ból~po~stracie, a jednocześnie złość, że nie mogłem nic zrobić”. Z drugiej strony, w niektórych przypadkach emocje są wyjątkowo wyraziste i jednoznaczne, zwłaszcza w tekstach nacechowanych emocjonalnie lub w krótkich, intensywnych wypowiedziach jak wybuchy gniewu -- „Nienawidzę tego!” czy wyrazy uwielbienia -- „Kocham to!”. W takich sytuacjach modele klasyfikacyjne mogą osiągać wysoką skuteczność bez potrzeby skomplikowanego przetwarzania kontekstowego~\cite{Emotions}.
\end{enumerate}

\section{Krótka historia rozwoju modeli NLP}
Analiza sentymentu i emocji w NLP ulega w ostatnich latach ciągłemu rozwojowi. Rozwój ten można podzielić na kilka etapów, odzwierciedlających postęp w metodach przetwarzania języka naturalnego. Początkowo, w latach 50. i 60. XX wieku, w NLP stosowano podejścia oparte na regułach i słownikach, takie jak tłumaczenie maszynowe przy użyciu prostych metod odwzorowania wyrazów~\cite{NLPHistory1}. W latach 80. XX wieku rozpoczęła się stopniowa transformacja w kierunku uczenia maszynowego -- dziedziny, która zastępowała ręcznie pisane reguły algorytmami uczącymi się na danych~\cite{NLPHistory2}. Wraz z rozwojem uczenia maszynowego oraz rosnącą dostępnością mocy obliczeniowej w latach 90. XX wieku zaczęto wykorzystywać modele klasyfikacyjne, takie jak \textit{Support Vector Machines} (SVM) oraz \textit{Naive Bayes}, które analizowały tekst na podstawie wyodrębnionych cech~\cite{NLP2017}.

Lata 2010. przyniosły kolejny postęp, kiedy to modele oparte na sieciach neuronowych, takie jak \textit{Long Short-Term Memory} (LSTM) oraz \textit{Gated Recurrent Units} (GRU), zaczęły być szeroko stosowane. LSTM, zaprezentowane po raz pierwszy w 1997 roku, zyskało na popularności właśnie na początku lat 2010. GRU, zaprezentowane w 2014 roku, stanowiło uproszczoną wersję LSTM, zapewniając podobne efekty, ale z mniejszymi wymaganiami obliczeniowymi. Dzięki zdolności do modelowania długoterminowych zależności w danych tekstowych, oba modele poprawiły jakość analizy sentymentu i emocji. Okazały się także skuteczne w przypadku dłuższych i bardziej złożonych tekstów~\cite{SentA2020}.

Kolejnym kamieniem milowym było wprowadzenie modelu BERT (\textit{Bidirectional Encoder Representations from Transformers}) przez firmę Google w 2018 roku. Model ten wykorzystuje mechanizm samouwagi (\textit{self-attention}), umożliwiający równoczesne uwzględnienie kontekstu zarówno poprzedzających, jak i następujących słów w zdaniu. Dzięki temu model potrafi uchwycić dwukierunkowe zależności między słowami, co znacząco poprawiło dokładność w zadaniach klasyfikacyjnych~\cite{BERT}.

Warto również wspomnieć o rozwinięciach modelu BERT, takich jak RoBERTa, ALBERT czy DistilBERT, które oferują różne kompromisy pomiędzy szybkością działania, rozmiarem modelu oraz efektywnością. RoBERTa poprawia jakość poprzez zmodyfikowany proces treningowy i większe zbiory danych~\cite{RoBERTa}, ALBERT redukuje rozmiar modelu poprzez współdzielenie parametrów~\cite{ALBERT}, a DistilBERT stanowi wersję uproszczoną, lżejszą i szybszą kosztem nieznacznej utraty dokładności~\cite{DistilBERT}. W niniejszej pracy zdecydowano się jednak na wykorzystanie oryginalnego modelu BERT ze względu na jego szeroką dostępność, ugruntowaną pozycję w~literaturze oraz umiarkowane wymagania zasobowe, które okazały się wystarczające dla badanego problemu. Ponadto, zastosowanie BERT-a umożliwia łatwe porównanie wyników z innymi badaniami.

W niniejszej pracy porównane zostaną wybrane modele uczenia maszynowego, które odzwierciedlają ewolucję podejść stosowanych w NLP na przestrzeni ostatnich dekad. Analiza obejmie zarówno klasyczne metody, jak i nowoczesne architektury oparte na sieciach neuronowych i transformatorach, umożliwiając ocenę ich skuteczności oraz efektywności w kontekście klasyfikacji sentymentu i emocji w tekstach.

\endinput